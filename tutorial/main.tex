% (zap "" "main")
\documentclass{article}

\usepackage{array}
\usepackage{verbatim}
\usepackage{listings}
\usepackage{pmboxdraw}
\usepackage[hidelinks]{hyperref}

\title{pbqff tutorial}
\author{Brent R. Westbrook}

\begin{document}

\maketitle
\newpage
\tableofcontents
\newpage

\section{Introduction}

pbqff is a program for making the generation of quartic force fields
(QFFs) as easy as possible. Its name was inspired by the C\&E News
article ``As DFT matures, will it become a push-button technology?''
by Sam Lemonick. The consensus among the Tschumper and Fortenberry
groups was that ``DFT'' or computational chemistry more generally
could never be fully ``push-button'' because of the need for expert
analysis of the data. Using pbqff is no different; you still need to
be that expert to make sure the output is reasonable and to write it
up into a paper. However, pbqff should make the experience of actually
running the QFF as painless as possible. If you have ever run a QFF
``by-hand,'' you know the complexity of the procedure. These tedious
and repetitive steps lend themselves perfectly to being done by the
computer, freeing you to think about something more interesting. If
you haven't, I will prepare a separate document giving a general QFF
tutorial. To understand what pbqff is doing, you should have a good
grasp on what \textit{you} would be doing if you had to do its job by
hand. It's not strictly necessary to understand pbqff though, and the
goal for me was really to make the implementation details irrelevant
to the end user. If you really want to understand the entirety of
pbqff and all its implementation details, you should just read the
source code, which can be found at
\href{https://github.com/ntBre/pbqff}{github.com/ntBre/pbqff}. Assuming
you want to get to work, I hope this tutorial document will give you
all the information you could possibly want for actually using pbqff
effectively. I will also prepare a separate man page to use for
quicker reference once you have a basic understanding.

\section{Mental Framework}

\subsection{Internal Coordinates}

As I alluded to in the introduction, the idea behind pbqff is to
follow the exact steps a person would by hand in running a QFF, but to
do so quickly and exactly via the computer. As it stands, I think of
pbqff as ``replacing'' an undergraduate researcher because we usually
give undergraduates all of the template input files they need. pbqff
requires that you give it template intder, spectro, anpass, and Molpro
files for a symmetry internal coordinate (SIC) run, which it can then
modify as needed. This modification assumes that the template input
files correspond to molecules with the same coordinate system as the
target molecule, so the automation at that stage is fairly
limited. Graduate students (and advanced undergraduates) usually learn
how to generate or more substantially modify intder, anpass, and
spectro files, so once pbqff can do that, it will have ``replaced''
them as well. Also as mentioned above, these replacements are not
total, as it still requires some thought to evaluate the program
output and to come up with new molecules for input. In a section on
mental models, think of the replacing as a rough model for how much
pbqff can do.

\subsection{Cartesian Coordinates}

The deal for Cartesian (or XYZ) coordinate QFFs is a little better
because intder and anpass are not used. Consequently, you only need
template spectro and Molpro files. This is also true for gradient
calculations since they are also based on Cartesian coordinates. Even
more fortunately, I think spectro input files are the easiest to
generate, so the earliest full automation will come to these types of
calculations as well. If you are familiar with our traditional SIC
QFFs, you may be surprised that intder and anpass are not used. Intder
is used for coordinate transformations from internal coordinates to
Cartesian coordinates and back again, so working directly in Cartesian
coordinates from the start obviates any need for it. Anpass is used
for least-squares fitting of the potential energy surface, whence we
gather the force constants. For the Cartesian QFF, pbqff numerically
computes each force constant directly, so there is no need to do a
fitting. These facts should not change your mental model of the whole
procedure, however. The big idea is still to take a geometry, displace
its atoms, compute single-point energies at each of the displaced
points, use those energies to obtain force constants, and then jam
those force constants into spectro to get spectroscopic data. In
Cartesian coordinates you just get to take shortcuts in the
displacement and force constant steps. Like with SICs, these are the
exact steps you would take in doing a Cartesian QFF by hand, but there
are so many points involved in Cartesian QFFs that we never do them by
hand.

\subsection{Job Submission}

The other slightly tricky part about pbqff if you are more used to our
conventional QFF schemes is that it does not submit, or even generate
input files for, all of the single point energy calculations at
once. Doing everything at once is obviously convenient when you are
doing it by hand because you can submit all of the computations to the
supercomputer and then go do something else. Instead, pbqff watches
the running jobs, continuously checking for ones that finish, and
writes and submits more calculations as the old ones finish. It also
deletes the files associated with the old jobs to save space. Space is
not typically a major concern with SIC QFFs since even the largest
systems we have worked with have only 10000 or so points. In contrast,
even water has half that many points in Cartesian coordinates, and
larger molecules that we've actually run have had over
200000. Eventually we would like to extend this to millions of points,
so keeping all of the files around is not really feasible. Again, you
should be able to picture this in the same way as doing it by hand,
but it's easier to convince the computer to sit and constantly refresh
qstat than to make a human do it.

A final wrinkle is the use of GNU parallel to reduce the number of
individual PBS files that need to be submitted. parallel is basically
a miniature version of the whole queuing system in that it takes a
list of jobs to run and dispatches them as resources are available on
a single node. While this is yet another departure from how you would
likely submit these jobs by hand, you can certainly use parallel when
submitting jobs by hand. It should also have no bearing on your
interaction with the program, but if you are curious why there are so
few jobs in your queue this is the explanation. As shown in
\autoref{tab:kwords}, I use the term ``chunk'' to refer to a group of
jobs submitted in one parallel PBS file, and the word ``job'' to refer
to the individual calculations that compose a chunk.

\section{Program Input}

In addition to the template files addressed in the previous and
upcoming sections, pbqff takes its own input file. This section will
walk through all of the accepted input options and offer some example
inputs for the various supported calculation types.

Unlike some programs, the input directives are totally
order-agnostic. As long as they are in the input file, they will be
recognized. Available keywords are shown in \autoref{tab:kwords}.  The
case of the keywords is ignored by the parser, but each keyword must
be followed by an equals sign (=). Comments can only start at the
beginning of a line, by including ``\#'' as the first character on the
line. Geometry input is unique in that it expects to look like
\verb!geometry={...}!, beginning with an equals sign and opening curly
brace and terminating with a closing curly brace. As shown in the
options for the GeomType, the geometry can be input as either a
Cartesian/XYZ geometry or a Z-matrix. If the Cartesian coordinates are
used, the program expects a fully-formed XYZ geometry, including the
number of atoms line and the comment line. These lines are skipped, so
it's not important that they be accurate, but they must be present.

The following examples can be found in my home area on hpcwoods under
Programs/pbqff/examples. The files embedded in this document should be
synced with the ones found there, and they all should be tested to run
correctly. 

\begin{table}[ht]
  \centering
  \caption{Keywords}
  \label{tab:kwords}
  \begin{tabular}{|l|>{\raggedright}p{0.3\textwidth}|l|p{0.6\textwidth}|}
    \hline
    Keyword                       & Type: Available values &                Default &                 Description \\
    \hline                       
    QueueType                     & String: sequoia, maple &                  maple & Specify which queue to target.
                                                                                      Basically choose the internal
                                                                                      PBS template to use for      
                                                                                      submitting jobs. \\          
    \hline
    Program                       &  String: cccr, cart, gocart,
                                    grad, molpro                  &                 molpro & Specify the subprogram
                                                                                             to use. cccr is for          
                                                                                             CcCR                         
                                                                                             SICs; cart or gocart is for          
                                                                                             Cartesians; grad is for      
                                                                                             gradients; and molpro        
                                                                                             is for normal SICs \\        
    \hline
    Queue                         &    String: workq, r410 &                   Both &               Select the queue
                                                                                      name to use. Default behavior
                                                                                      is to use whatever is             
                                                                                      available.\\                 
    \hline
    Delta                         & Float: any & 0.005 & Specify the step size to use for the
                                                         geometry displacements of a Cartesian QFF. \\
    \hline
    Deltas                        & Int:Float pairs: any & 0.005 & Specify step sizes for individual coordinates
                                                                   in a Cartesian QFF.
                                                                   Format is index:value, where index starts from 1.
                                                                   Pairs are separated by commas. Any indices not
                                                                   specified take their default value from Delta.\\
    \hline
    GeomType                      & String: xyz, zmat & zmat & Specify the type of geometry.
                                                               Currently this is only used to
                                                               compute the number of coordinates.\\
    \hline
    Flags                         & String: noopt & None & Specify command line flags in the input file.
                                                           Only noopt is currently supported.\\
    \hline
    Deriv                         & Int: 2, 3, 4 & 4 & Specify the derivative level to be computed for a Cartesian QFF.\\
    \hline
    JobLimit                      & Int: any & 1000 & Specify the maximum number of jobs to have submitted at once.\\
    \hline
    ChunkSize                     & Int: any & 64 & Specify the number of jobs to submit in a GNU parallel ``chunk.''
                                                    This also determines how often files are deleted.\\
    \hline
    CheckInt                      & String/Int: no/any & 100 & Specify the checkpoint interval. An input of ``no''
                                                               disables checkpointing, while an integer value sets
                                                               the interval.\\
    \hline
    SleepInt                      & Int: any & 1 & Specify the interval at which to poll running jobs in seconds.\\
    \hline
    NumJobs                       & Int: any & 8 & Specify the number of jobs to run per GNU parallel chunk.
                                                   Each parallel job requests 64gb of memory, so NumJobs
                                                   should evenly divide 64. This is also how many CPUs are requested,
                                                   so it should not exceed the maximum available.\\
    \hline
    IntderCmd                     & String: any & None & Specify the path to the intder executable for SIC QFFs.\\
    \hline
    AnpassCmd                     & String: any & None & Specify the path to the anpass executable for SIC QFFs.\\
    \hline
    SpectroCmd                    & String: any & None & Specify the path to the spectro executable.\\
    \hline
    Geometry                      & String Block: any & None & Specify the geometry. See the examples for details.\\
    \hline
  \end{tabular}
\end{table}

\subsection{SIC Example}

Below is an example input file for an SIC run. Since much of the
information for the SICs is found in the other template input files,
the pbqff file is about as minimal as it gets. Based on the defaults
given in \autoref{tab:kwords}, even the \verb|program|,
\verb|queue|, and \verb|geomType| lines are technically redundant, but
it's nice to include some of these for future reference. Example:

\lstinputlisting[
caption={pbqff input for an SIC calculation}
]{../examples/sic/sic.in}

One thing to note is that there is no brace between the Z-matrix
itself and the values of the parameters, as you might expect if you
are used to Molpro. Just remember that this is the format expected by
pbqff, and it will convert it to the Molpro format when necessary.

\subsection{Cartesian Example}

The Cartesian example is a bit more involved since you have to specify
some non-default values, and you can't rely on the intder file for the
step sizes or derivative level. Again, many of these options are
technically optional since they are the same as the defaults, but it's
nice to be explicit when possible. Example:

\lstinputlisting[
caption={pbqff input for a Cartesian calculation}
]{../examples/cart/cart.in}

Of note here is a first example of an XYZ geometry. You can see that
the number of atoms line and comment are present. Alignment and
spacing are not important, so you can freely paste the geometry in
however you think looks best. Another important aspect of this example
is the demonstration of the \verb|deltas| input. In this example, all
of the steps in the $x$ direction will be larger (of size 0.075
\AA{}), while the rest will be 0.005 \AA{}.

\subsection{Gradient Example}

The gradient version is virtually identical to the Cartesian version,
except that the \verb|program| is specified as \verb|grad|. You may
also notice that the comment line says that the reference energy was
computed at the DF-CCSD(T)-F12 level rather than regular
CCSD(T)-F12. Molpro only has analytic gradients for density-fitted
coupled cluster, so if you want to use gradients keep that in
mind. Based on some forthcoming research from our group, you probably
don't want to use gradients though. Example:

\lstinputlisting[
caption={pbqff input for a gradient calculation}
]{../examples/grad/grad.in}

\section{Template Files}

As mentioned previously, pbqff still requires you to give it template
input files for Molpro and spectro, in the case of Cartesian QFFs, and
for Molpro, spectro, intder, and anpass for SICs. It should probably
take template PBS files as well, but the flexibility therein has been
less necessary so far. The Molpro file is the most important because
it is required to run any computations, so we will start with that
one. Next, we will look at the spectro input file since that is common
to both Cartesian and SIC QFFs. However, the spectro input file is not
technically required for the program to run usefully. The final real
products of pbqff are the force constant files that are fed into
spectro. The program will print an error if it can't find the spectro
input file, but you can easily run spectro yourself (or with my
gspectro program) afterward. For SICs both the intder and anpass files
are strictly required by the program to even start running.

\subsection{Molpro}

The basic Molpro template file is very straightforward. The only
difference indicating that it is a template is the lack of a closing
brace in the geometry section. Other than that, you can include any
option accepted by Molpro and it will be transmitted directly into
each of the single point energy input files. The \verb|optg| line is
optional if you are going to use the \verb|noopt| flag anyway, but if
you use that flag the program will automatically remove that line from
the input.

\lstinputlisting[
caption={Basic Molpro template file example}
]{../examples/sic/molpro.in}

For a gradient calculation, a couple minor changes are required, as
shown in the listing below. In this case, the gradient is what needs
to be extracted from the output file, not the energy. Molpro prints
the energy with sufficient precision by default, but it only prints
the gradient to 8 decimal places. As a result, we need to use the
\verb|varsav| directive to save the gradients in each direction to
variables, and then the \verb|show| directive to print each of them in
a 20.15 format. The 20 is actually irrelevant, and 12 decimal places
should be sufficient, but you might as well request a little extra
precision just to be safe. Additionally, as touched on before, Molpro
only has analytic energy gradients for DF-CCSD(T), so make sure to use
both DF-HF and DF-CCSD(T) for the energy calculations.

\lstinputlisting[
caption={Molpro template file example for a gradient calculation}
]{../examples/grad/molpro.in}

No additional convergence criteria are specified in this gradient
listing, so make sure you include those if you are doing
publication-quality research. The typical criteria used in our papers
are demonstrated in the SIC listing.

\subsection{spectro}

The spectro input file should look like the listing below and must be
named spectro.in. The necessary sections are the initial
\verb|# SPECTRO ####| section, which includes the input directives;
the \verb|# GEOM ####| section, which has the geometry; the
\verb|# WEIGHT ####| section, which obviously has the masses; and the
\verb|# CURVIL ####| section, which contains the curvilinear
coordinate system to be used for the rotational and geometry
analysis. No resonance information should be included, and
correspondingly all of the input directives that enable resonance
accounting should be set to zero. I plan to have the program enforce
this in the future, but for now you need to make sure of it. The
geometry does not need to have the right geometry or even the same
atoms as the input molecule because the program will rewrite the file
with the appropriate values. However, the order of the atoms should
match the order present in the input geometry. If you are given a
template spectro file, I would advise changing the geometry in the
pbqff input file to match the spectro order. If instead you generate
the spectro file yourself, just make sure it lines up with the input
file. The weight section does not need to be correct unless you enable
its use in the input directives, as is usually the case with
spectro. It may not even have to be present, but that is a spectro
question, not a pbqff question.

{\tiny
\lstinputlisting[
caption={Example spectro input file for water}
]{../examples/grad/spectro.in}
}

Apologies for the size, but those integer input parameters take up a
lot of room! Just like gspectro, when pbqff runs spectro, it actually
runs it twice to incorporate the resonances on the second pass. That's
why it's safe to leave out the resonance information in the template
file. The input directives will also be updated to reflect the
necessary resonances, so taking your results from spectro2.out will
give the correct values. However, pbqff cannot yet handle degenerate
modes, at least in an explicit sense, so if you need to include a
\verb|# DEGMODE ####| section, it might work, but you should
double-check that part by hand. If you can confirm one way or the
other, please let me know.

\subsection{intder}

The listing below shows the top portion of a template intder file,
which must be named intder.in, but it shouldn't be too
interesting. pbqff just expects a normal intder file for generating
points, but it will take the geometry you leave in there and try to
match the optimized geometry to it by exchanging rows and
columns. This typically works very well if the intder file is from a
molecule with the same symmetry. If you encounter an error saying
``transform failed'' this coordinate transform is what it is referring
to. In that case, jump to the Troubleshooting section for a discussion
of how to address that using a couple of command line
flags. Otherwise, just feed it a full intder file, either one you
obtained from someone else or a previous calculation or one you
generated using taylor.py and its ilk.

{\tiny
\lstinputlisting[
caption={Partial example intder input file for water},
lastline=17
]{../examples/sic/intder.in}
}

\subsection{anpass}

Like that of intder, the anpass handling is very
straightforward. pbqff will parse the anpass file to make sure it
prints the correct number format and then replace the relative
energies at the ends of the lines in that section. It does not do any
validation of the step sizes or of the unknown blocks at the end, so
make sure you use a well-formed anpass input file that corresponds to
your intder file. Similar to spectro, pbqff will run anpass once to
find a new stationary point, and then run again at that stationary
point, just like you would do with the whole ``long-line'' thing by
hand. The file must be named anpass.in, which may require a quick
rename if you copy over an existing anpass1.in file from another
project. See below for an abbreviated listing of the top of the file,
just to be consistent with the other sections. Hopefully you know what
it's supposed to look like. The title only agrees with the contents of
the file because I got this from someone else. Rest assured that it
has no effect on the calculation.

\lstinputlisting[
caption={Partial example anpass input file for water},
lastline=12
]{../examples/sic/anpass.in}

\section{Running the Program}

Now that you have all of the requisite input files for any type of
calculation, you are ready to run the program! In this section I will
show the basic input command, along with an explanation of the parts
that aren't as basic as you would expect, and also describe each of
the flags you can use to modify the program's behavior. The most basic
form of the command is

\begin{verbatim}
$ pbqff infile.in
\end{verbatim}

\noindent
assuming that you have added the executable to your path under the
name pbqff. Since most of the interesting jobs you will be running will
take a while, you likely want to run

\begin{verbatim}
$ pbqff infile.in & 
\end{verbatim}

\noindent
instead, to put the job in the background. However, as you may or may
not know, logging out of your terminal session, i.e. exiting your ssh
connection, sends SIGHUP to all of the programs you had running. This
means in effect that any background task you started is killed when
you log out. To combat this, you need to tell the shell not to send
SIGHUP using the \verb|disown| command with its \verb|-h| flag. With
this in mind, the command I use to run pbqff is

\begin{verbatim}
$ pbqff infile.in & disown -h
\end{verbatim}

Assuming all goes well, this should immediately create infile.out and
infile.err. Currently, the .out file does not have too much
information in it, but the .err file reports the progress of your
calculation once the points start running. To follow these files, I
typically use the command

\begin{verbatim}
$ tail -F infile.out infile.err
\end{verbatim}

\noindent
where the \verb|-F| flag is like the lowercase version, but it will
pick up a file if it is newly created. This is not that important for
the out and err files, but if you also want to follow either the
optimization file or the reference energy file in the case of a
Cartesian run, it can come in handy. These two files are
\verb|opt/opt.out| and \verb|pts/inp/ref.out|, respectively.

\subsection{Directory Structure}

These filenames give a good lead-in to the directory structure created
by the program. In the case of SIC QFFs, I used my conventional
directory structure that looks like

\begin{verbatim}
.
├── freq
├── freqs
├── opt
└── pts
    └── inp
\end{verbatim}

\noindent
where opt is used for the geometry optimization, freq is used for the
Molpro harmonic frequency calculation, pts is used to run intder,
while pts/inp holds the actual input files, and freqs is where I run
intder, anpass, and spectro to get the anharmonic data. Cartesian QFFs
are more streamlined, so everything there is either run in the base
directory where pbqff is run or in pts/inp. In both cases, you
generally should not need to descend into the created directories
unless you are doing some heavy troubleshooting. The only real
exception is the SIC freqs directory, which will contain your final
spectro2.out file. Another less important exception could be the SIC
freq directory if you need to compare the harmonic frequencies to
those from intder and spectro in the freqs directory.

\subsection{Flags}

You can already access the main functions of pbqff and control its
behavior to some extent using the input file. However, there are also
some additional command line flags that you can use to further modify
the behavior. The most important flag is the \verb|-h| or \verb|-help|
flag because it will list the help for all of the other flags. I also
plan to replicate much of the fundamentals of this section in the man
page, but here I hope to give some examples and explanations of the
flags in addition to their basic usages. Like the keywords, the flags
can be found in \autoref{tab:flags}.

\begin{table}[ht]
  \centering
  \caption{Command line flags}
  \label{tab:flags}
  \begin{tabular}{|c|c|c|p{0.6\textwidth}|}
    \hline
    Flag & Type & Default & Description \\
    \hline
    c & Bool & False & Resume from checkpoint; requires the o flag to overwrite existing directory.\\
    count & Bool & False & Read the input file for a Cartesian QFF,
                           print the number of calculations needed, and exit.\\
    cpuprofile & String & None & Write a CPU profile to the supplied filename.\\
    debug & Bool & False & Print additional information for debugging purposes.\\
    fmt & Bool & False & Parse existing pts output files and print them in anpass format\\
    freqs & Bool & False & Start an SIC QFF from running anpass on the pts output.
                           This requires that all of the points have been preserved.\\
    irdy & String & None & Ignore the geometry in the input file and use the intder.in file
                           as is. This implies that you should use noopt. The string is a
                           space-delimited list of atomic symbols to pair with the geometry.\\
    nodel & Bool & False & Preserve output files instead of deleting them after use.\\
    o & Bool & False & Allow existing directories created by the program to be overwritten.\\
    pts & Bool & False & Resume an SIC QFF by generating the points from
                         the optimized geometry in the opt directory.\\
    r & Bool & False & Read the reference energy for a Cartesian QFF from
                       an existing pts/inp/ref.out file.\\
    \hline
  \end{tabular}
\end{table}

A couple important things to mention are the artifacts of the Go flag
package. Unlike most (if not all) Unix utilities, the Go flag package
does not support joining multiple flags together as in

\begin{verbatim}
$ ls -lt
\end{verbatim}

\noindent
which is equivalent to 

\begin{verbatim}
$ ls -l -t
\end{verbatim}

This means that when you want to resume a calculation from a
checkpoint file, which requires that you assent to overwriting
portions of the inp directory, you have to pass both the \verb|-c| and
\verb|-o| flags as in

\begin{verbatim}
$ pbqff -o -c infile.in & disown -h
\end{verbatim}

What this does afford, however, is the ability to use only a single
dash for long options or equivalently to use flags longer than single
letters. That's how and why many of the flags are more descriptive
than they might have been otherwise. If you are already in the habit
of using double dashes for long options, it will also handle
that. Similarly, you can join arguments to flags using an equals sign
as in

\begin{verbatim}
$ pbqff -irdy="H O H"
\end{verbatim}

\noindent
or without one as in

\begin{verbatim}
$ pbqff -irdy "H O H"
\end{verbatim}

\noindent
but you cannot join the argument directly to the flag since it could
be parsed as a single-dash long option. For the currently-implemented
flags, this isn't really a problem, but I figured it doesn't hurt to
point out.

\subsection{Checkpoints}

Since most interesting calculations are going to take a long time to
run, the program comes equipped with a checkpointing system to save
the progress of a run as you go. The JSON files created while the
program is running serve as these checkpoints, at least for Cartesian
QFFs. Checkpoints are not currently implemented for SICs, so if you
are particularly worried about an SIC calculation, I would suggest
using the \verb|nodel| flag since it's more straightforward to resume
an SIC as long as all of the output files are present. When resuming
from a checkpoint, the program behaves mostly as if it were starting
from scratch, but instead of recalculating all of the energies, it
will load as many as possible from the JSON files. I already showed an
example of resuming from a checkpoint above, but just to keep it
explicit here it is again:

\begin{verbatim}
$ pbqff -o -c infile.in & disown -h
\end{verbatim}

The \verb|-o| flag is required since you are going to have to
overwrite some of the previous files, and the \verb|-c| flag triggers
the checkpoint loading. You should run this command in the directory
where you initially ran pbqff. This should be fairly obvious since
that's where your input file is.

When might you need a checkpoint? The most likely scenario on Maple is
if your job runs out of CPU time. If you run the command

\begin{verbatim}
$ ulimit -aH
\end{verbatim}

\noindent
you will get a list of your hard resource limits on Maple. At the time
of this writing, the CPU time limit is 72000 seconds or 20 hours. Of
course, CPU seconds are not equivalent to wall seconds, so this does
not mean you can only run jobs that finish within 20 hours. What it
does mean though is that if pbqff exceeds that usage it will be killed
automatically. The easiest way to identify that this has happened is
by checking the modification time of the err file since it should be
updated every second. You can also run the command

\begin{verbatim}
$ ps aux | grep pbqff
\end{verbatim}

\noindent
to check for a running process with the name pbqff. Alternatively you
can grep for your username or your username and pbqff to further
narrow the results. pbqff will print its CPU usage at every checkpoint
interval, but if you want a more regular way to check, you can use the
command

\begin{verbatim}
$ ps axo pid,user,comm,time | grep pbqff
\end{verbatim}

\noindent
The important part of this command is the \verb|time| portion, but the
other fields requested are useful in case you want to kill the program
(pid), and for sorting based on username and process name (user and
comm). If you check this and see your pbqff instance approaching 20
hours, expect to have to restart it soon. Since pbqff doesn't know
when it will be killed, it will likely have started many jobs already
that will only be in the way of your new instance. As a result, you
should kill all of the jobs in your queue using a command like

\begin{verbatim}
$ qselect -u $USER | xargs qdel
\end{verbatim}

\noindent
The \verb|qselect| command is used to select jobs in the queue based
on the arguments. The \verb|-u| flag selects by user, and the
environment variable \verb|USER| is your username. You can also select
by name using the \verb|-N| flag. This can be more useful if you have
jobs other than those associated with pbqff in the queue and you don't
want to kill them. The name of the single point energy jobs submitted
by pbqff is ``pts,'' giving the command

\begin{verbatim}
$ qselect -u $USER -N pts | xargs qdel
\end{verbatim}

\noindent
if you want to be a bit more selective.

Another common issue that will create a huge error dump in the err
file is using too many operating system threads. The number of threads
available is much more limited than even the CPU time, so if you try
running more than one instance of pbqff at a time, both will likely
crash with this error. The same can be true of any intensive
action. As a result, while pbqff is running, you should be careful to
limit your resource usage on Maple. Obviously any interactions with
the actual queue need to be done on Maple, but if you need to do some
kind of file scripting or even just tailing the pbqff output and error
files, you can do that on Woods instead.

\section{Troubleshooting}

The most common problems have already been addressed in the previous
section since they will require a restart from the checkpoints. In
this section, I will describe other potential problems you may
encounter and offer advice on how to solve them. If you run into any
problems not covered here, please reach out to me and I will add it to
this section.

\subsection{Transform Failed}

If you get this error message in an SIC QFF, it means that the program
was unable to match the optimized geometry of your molecule to the
pattern it identified in the intder template file. Assuming you have
an actually usable intder template, this can be caused by slight
symmetry differences in the two geometries that are not ignored by the
fairly simple pattern matching algorithm. You may be able to recognize
this and modify the template slightly to allow the program to match
the pattern automatically, but this is likely difficult to do without
knowing the implementation details of the matcher. An easier option is
to manually take the optimized geometry and place it in the intder
template file as if you were going to run the QFF by hand. From there
you can tell the program that the intder file is ready by using the
\verb|-irdy| flag. The string argument to this flag is a
space-delimited list of the atoms in the geometry. Usually the program
gets the order of the atoms from the optimized geometry and keeps
track of their order through the pattern matching. Without that, it
needs you to tell it which atom is which. As shown above, the full
command to restart from this point is

\begin{verbatim}
$ pbqff -o -irdy "H O H" & disown -h
\end{verbatim}

\noindent
where the overwrite flag is necessary if you have already run the
program in the same directory. You will also probably want to add the
line \verb|flags=noopt| to your input file since you will have
obviously already optimized the geometry to put it in the intder file.
That's not necessary for the program to function, but it will save you
the time of waiting for it to optimize again.

% sic problem matching pattern - set up intder yourself, pass noopt and irdy

\end{document}